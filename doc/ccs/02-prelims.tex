\section{Preliminaries} \label{sec:preliminaries}
In this section we give a more detailed overview of the properties of the NDN architecture relevant to the issue of anonymous communication. We then also delve into the design of AND\=aNA (henceforth referred to as AND\=aNA-v1), to identify the major security flaws and engineering shortcomings that are remedied in AND\=aNA-v2. 

\subsection{NDN Overview}
Named-data networking (NDN) is one of several proposed information-centric network designs under active research and development as the future Internet architecture. Its defining characteristic is that it decouples the location of data from its original publisher. All requested content is cached in network-layer routers between consumers who request data and publishers, or other routers, satisfying such data. In essence, network caches and addressable content, rather than addressable hosts or interfaces, enable this new architecture to reduce network congestion and latency by keeping content closer to its intended recipients. 

Content is requested through the issuance of an \emph{interest}, or a meaningful URI with a one-to-one correspondence with content provided by the network. The components of an interest are arbitrary strings and can therefore be used to store any type of data, including human readable names or binary data encoded as URI-friendly strings. Upon receiving an interest, a router looks for a match in its \emph{content store} (CS), which is the cache that stores content already requested from other consumers. Matching is done using exact-match based on content names, i.e., a complete interest name match in the CS will cause the associated content to be forwarded downstream on the same router interface upon which the interest arrived. Interests that do not completely match any content name in the CS are stored in a \emph{pending interest table} (PIT) together with the corresponding interface upon which the interest arrived and is subsequently forwarded to the appropriate upstream router based on contents in a \emph{forward interest base} (FIB) table. Multiple interests matching the same name are collapsed into a single PIT entry to prevent redundant interests being sent upstream. Once a content matching a PIT entry is received by a router, the content is cached (unless the interest has explicitly marked the content to not be stored) and sent to all downstream interfaces associated with the PIT entry. Finally, upon completion, the PIT entry is cleared.

Content-centric traffic also has strong security implications. Firstly, it means that content security is tied to the data itself, not the channel through which it flows between consumers and the network. All sensitive content must therefore be protected with a suitable form of encryption to ensure confidentiality. Content integrity is guaranteed by digital signatures; all content producers are required to sign content before responding to incoming interests. Due to performance impediments, signature verification is not required by network routers; naturally, consumer applications are expected to verify content signatures. Issues regarding signature verification and key management are discussed more at length in \cite{}.Secondly, a lack of source and destination addresses improves user privacy. However, as we will discusse in the following section, this lack of information is insufficient with regards to consumer privacy. 

\subsection{AND\=aNA-v1 Design Highlights and Pitfalls}
AND\=aNA was originally designed as a proof-of-concept equivalent of Tor \cite{Tor} in the context of NDN. 
