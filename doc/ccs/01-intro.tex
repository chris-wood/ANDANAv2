\section{Introduction} \label{sec:introduction}
Usage of the Internet has undergone a tremendous trasnformation since its inception in the 1970s. Content distribution, as opposed to point-to-point communication, has become the leading type of traffic traversing today's valuable network resources; Netflix, for example, accounted for nearly 30\% of all downstream traffic in 2012 \cite{Netflix}. The number and popularity of such information-centric services are only expected to increase in the future with the growing presence of data-intensive consumer applications and devices (e.g., media streaming applications and mobile devices), leading to added pressure on network resources and a subsequent increase in network congestion and wasted bandwidth. 

Named-data networking (NDN) \cite{ndn-techreport} is an emerging network architecture capable of supporting information-centric traffic. Two primary characteristics of the NDN architecture are that content names, rather than hosts or locations, are addressible and routable through the network, and all generated content corresponding to some name must be signed by its original producer. The latter property decouples content confidentiality, integrity, and authenticity from content and the manner in which it is delivered to consumers (e.g., instead of using secure tunnels akin to SSH/TLS, content can be encrypted before sent to a consumer). These fundamental design decisions enable content to be cached in network-layer resources throughout the network, thus promoting reduced network congestion and wasted bandwidth when popular content is requested. 

The NDN design builds content security support \emph{into the architecture} and promotes content access via producer-specified forms of encryption. Similar to the IP-based counterpart, consumer and producer anonymity, however, are not readily supported by this design. While there are no longer source or destination addresses associated with traffic, there are a variety of other sources of information via which consumers can be deanonymized, including: interest and content names, network router cache contents, and digital signature contents. An intelligent adversary may use any of these information sources when launching client deanonymization attacks. 

{\sf AND\=aNA}, an anonymous named-data networking application, pioneered support for anonymous content retrieval and distribution with regards to consumers and producers, respectively \cite{andana}. Inspired by Tor \cite{Tor}, {\sf AND\=aNA} uses onion-encryption to wrap requests for content that are iteratively decrypted and forwarded by participating anonymizing routers, and also to wrap content as it flows from the producer to the consumer. Unlike Tor, however, {\sf AND\=aNA} was a proof-of-concept application-layer anonymizing layer for NDN that targeted unidirectional traffic \emph{without} real-time requirements. Support for high-throughput, low-latency, bidirectional traffic, such as traffic generated from voice communication, video chat, and media streaming applications. In addition to this severe performance impediment, the ``optimized'' session-based variant of {\sf AND\=aNA} sacrificed interest and content linkability (an anonymity issue) for only slightly improved efficiency. 

Consequently, in this work we present a new and substantially improved design for an NDN anonymizng application, henceforth referred to as {\sf AND\=aNAv2}, that addresses the many performance and anonymity pitfalls of the original design. We discuss the design and implementation of {\sf AND\=aNAv2} as built upon CCNx \cite{ccnx}, the open-source relative of NDN. We also present performance results from testing our application implementation in numerous test environments with different types of traffic. Our results show that our modified design leads to  substantial performance gains without sacrificing consumer or producer privacy, thus making {\sf AND\=aNAv2} a promising application to support a diverse set of application content and use cases in future information-centric networks.
