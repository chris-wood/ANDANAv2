\section{Implementation} \label{sec:introduction}
{\sf AND} is implemented entirely in C using the CCNx 0.82 library \cite{ccnx}. Anonymizing routers run instances of the {\tt AnonServer} application with a single prefix URI, whereas consumers run instances of the {\tt AnonConsumer} application with a set of configuration options and sequence of prefix URIs denoting the circuit to be constructed. The {\tt AnonServer} module instantiates a single {\tt DownstreamProxy}, which is a subclass of the base {\tt Proxy} class used to interface with the CCN daemon running on the host. This downstream proxy is responsible for unwrapping incoming interests and wrapping upstream content. It is also responsible for maintaining the session state information associated with each circuit, which includes populating the state table for both the handshake and online session establsihment variants. 

The {\tt AnonConsumer} application creates a circuit of {\tt UpstreamProxy} instances, which are also subclasses of the base {\tt Proxy} class, that are responsible downstream content encryption, as well as a single {\tt UpstreamProxy} that is responsible for generating the wrapped version of each incoming client interest. Upon instantiation, each {\tt UpstreamProxy} establishes session state and either persists it for the online session establishment or shares it with the corresponding {\tt DownstreamProxy} instance running on the anonymizing proxy if the handshake variant is used. 

The majority of the functionality is encapsulated in the four procedures, {\tt WrapInterest}, {\tt UnwrapInterest}, {\tt WrapContent}, and {\tt UnwrapContent}. Additional functionality, such as interest window maintenance (see Section \ref{sec:windowing}), is implemented by the respective proxy classes in conjunction with helper classes such as {\tt ProxyState}. 
